1\section{Map representation}
The particular Sokoban puzzle that must be solved by the robot, is handed out beforehand in a textual format. An example of this textual format is:
\\\\
XXXXX\\  
X......XXXXX\\ 
X..J..J..J...X\\  
XXX..X..X...X\\   
\hspace*{6 mm}X..X.....X\\  
\hspace*{3 mm}XX..XXX..XX\\        
\hspace*{3 mm}X..GGG..G..X\\ 
\hspace*{3 mm}X....MG..J..X\\  
\hspace*{3 mm}X..XXX..XXX\\     
\hspace*{3 mm}X............X\\
\hspace*{3 mm}XXXXXXX  
\\\\ 
where 'X' represents 'Wall' squares, '.' represent 'Emtpy' squares, 'J' represents 'Jewel' squares, 'G' represents 'goal' squares and finally 'M' represents the 'Man' square. 

This map representation will be converted directly into a multidimensional array, in order to use the map in graph traversal by the chosen search algorithm. The multidimensional array representation is preferable, because a particular squares x and y coordinate translates directly to the corresponding row and column of the multidimensional array, which in turn enables the algorithm to quickly look up possible routes.