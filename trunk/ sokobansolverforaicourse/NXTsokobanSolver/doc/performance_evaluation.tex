For each search algorithm, three consecutive runs have been performed with each search algorithm, and the data from the best run chosen.\\

\begin{table}[h!]
	\centering
	\begin{tabular}{| l | c | c | c | c | }
		\hline
			Search algorithm	& States traversed & Memory used [MB] & Robot moves & Time taken [s]\\ \hline
	    	Uniform Cost & 18968 			& 39 		 & 142 	& 202\\\hline
		    Breadth first (Non unit-step)	& 19761 			& 40 		 & 162 	& 218\\\hline
		    A*		& 18788 			& 39 		 & 142 	& 189 \\\hline
	    	Greedy 		& 18215 			& 48 		 & 212 	& 179 \\
		\hline
	\end{tabular}
	\caption{Results from the different search algorithms.\label{tbl:searchresults}}
\end{table}
Another search has been performed with the tree pruning algorithm disabled (No removal of deadlock states), to see how big an effect the pruning has on the performance of the solver. The search algorithm used is A*, because it outperformed the other search algorithms.\\
\begin{table}[h!]
	\centering
	\begin{tabular}{| l | c | c | c | c | }
		\hline
			Search algorithm	& States traversed & Memory used [MB] & Robot moves & Time taken [s]\\ \hline
		    A*		& 148097 			& 318 		 & 142 	& 12223 \\
		\hline
	\end{tabular}
	\caption{Results from the different search algorithms, when search tree pruning is disabled.\label{tbl:searchresultsnoprun}}
\end{table}


\subsection{Planner conclusion}
As can bee seen from the Uniform Cost search in table \ref{tbl:searchresults}, the optimal solution contains 142 robot movements (102 if the solver did not take into account the number of moves that the robot must perform in reality). As predicted the breadth first and greedy searches do not find the optimal solution, and would therefore not be applicable to use for this particular project, even though greedy search outperforms the other searches in terms of speed. As it turns out the A* search outperforms the Uniform Cost Search in number of states traversed and time taken, which was not all that unexpected. It can therefore be concluded that the sum of the manhatten distances between the each jewel and its closest goal, is a heuristic that will yield an optimized search. It can furthermore be seen from table \ref{tbl:searchresultsnoprun}, as expected that the effects of pruning far outweigh the gains that one particular search algorithm provides over another in terms of time taken to find solution, the states traversed and the memory used.
